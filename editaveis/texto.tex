\chapter{Introdução}\label{introducao}
% determinacao da potencia sonora
A potência sonora é uma grandeza física muito importante para caracterizar uma fonte sonora. Para mensurar tal característica usa-se ambientes reverberantes e anecóicos, usando equipamentos específicos para cada contexto afim de usar parâmetros desses ambientes na determinação da potência sonora. Foram então usadas as câmaras semianecóica e reverberante para mensurar o nível de potência sonora e ao final do processo os resultados foram comparados com os valores tabelados da fonte. Concluiu-se que os ensaios refletiram o que era esperado através da comparação com os resultados tabelados, ou seja, as potências sonoras das câmaras anecóica e reverberante seguiram os resultados tabelados com a câmara reverberante o mais próximo do resultado ideal.

\chapter{Fundamentação Teórica}\label{fundamentacao}
% Som
Em vista do que se expõe em \cite{bistafa}, o som é a aceleração de partículas de ar que se chocam, transformando a pressão estática do ar numa pressão oscilatória. Essa pressão oscilatória entre em contato com os ouvidos do ouvinte fazendo com que essa variação de pressão seja percebida e escutada. Essa variação de pressão que oscila é escutada pelo ser humano a partir da magnitude de pressão de $2\cdot10^{-5}$ $Pa$. Qualquer pertubação de pressão que se encaixa nessas condições é denominada som e as pertubações indesejáveis são denominadas de ruído sonoro.

Para se mensurar o nível de pressão sonora deve-se comparar	o valor $rms$ da pressão coletada pela pressão de referência ($2\cdot10^{-5}$ $Pa$), usando a operação de divisão. Como a magnitude dessa divisão é muita alta, a variação das pressões audíveis é de ordem muito alta fazendo assim que se transforme esse valor numa medida logarítmica e multiplicado por 10. A expressão matemática resultante desse processo é 
\begin{equation}
  NPS  = 10 . log_10(p_{rms}^{2}/(2\cdot10^{-5})^{2}).
\end{equation}
Tal que $NPS$ é o nível de pressão sonora e $p_{rms}$ é o $rms$ da pressão coletada.
% pressao sonora

% potencia
De acordo com \cite{potencia}, a potência sonora é uma grandeza física que diz respeito à energia acústica total emitida por uma determinada fonte sonora. Dessa forma, a potência sonora depende apenas da própria fonte e independe das características do meio, fazendo com que esse tipo de grandeza física se qualifique como satisfatório para caracterizar uma fonte sonora. Em vista disso a potência sonora é dada pela equação
\begin{equation}
W = I_{max}S.
\label{eq.potencia}
\end{equation}
Tal que $W$ é a potência sonora, $I_{max}$ intensidade sonora máxima e $S$ é a área. Como a intensidade sonora máxima ($I_{max}$) se relaciona com a impedância característica do meio ($\rho . c$) de tal forma que a expressão matemática de denota como \begin{equation}
I_{max}=\frac{p_{rms}^{2}}{\rho_{0} . c}.
\label{eq.intensidade}
\end{equation}
Dessa forma, dividindo por uma potência de referência $W_{o}= \frac{p_{o}^{2}}{\rho c}S $ a equação da potência pode ser caracterizada por
 \begin{equation}
	\frac{W}{W_{o}}=\frac{p_{rms}^{2}}{p_{o}^{2}}\frac{S}{S_{o}}.
\label{eq.relpotencia}
\end{equation}
Aplicando a operação logarítmica na equação \ref{eq.relpotencia} o resultado final do cálculo da potência é
\begin{equation}
	NWS = NPS + 10 . log_{10}\left(\frac{S}{S_{o}}\right).
\end{equation}

% calculo potencia anecoicaa
Para se medir a potência sonora há ambientes apropriados que são as câmaras anecóicas e câmaras reverberantes. De acordo com \cite{bistafa} as câmaras anecóicas, são construídas com superfícies configuradas para absorver toda a energia sonora incidente, simulando um campo livre com reflexões anuladas nas paredes. No extremo oposto, as câmaras reverberantes são construídas de tal forma a maximizar o som refletido pelas paredes, no sentido de gerar campo difuso. As imagens \ref{figura_1} e \ref{figura_2} ilustram exemplos de câmara anecóica e reverberante respectivamente.

\begin{figure}[h!]
    \centering
    \includegraphics[width=0.9\textwidth]{figuras/P_anecoica.eps}
    \caption{Câmara semianecóica. Fonte: http://lva.ufsc.br/}
    \label{figura_1}
\end{figure}

\begin{figure}[h!]
    \centering
    \includegraphics[width=0.9\textwidth]{figuras/P_anecoica.eps}
    \caption{Câmara reverberante. Fonte: http://lva.ufsc.br/}
    \label{figura_2}
\end{figure}

Para medir a pressão sonora nas câmaras anecóicas deve-se usar os microfones de campo livre nos quais são projetados de forma direcional para a fonte sonora. Essa caracterísca faz a onda incidir de forma longitudinal no microfone. Para se medir a pressão sonora nas câmaras reverberantes deve-se usar os microfones de campo difuso nos quais são projetados de forma que as ondas se incidem em todos os lados desse tipo de microfone.

% calculo potencia reverberante

\chapter{Experimento e Equipamentos}\label{descricao}

O objetivo principal desse experimento é mensurar o nível de potência sonora de um fonte sonora nos ambientes de câmara anecóica e reverberante. Ao final deve-se comparar os resultados obtidos com os tabelados do aparelho.

% instrumentos de medicao
Para realizar as medições tais instrumentos foram utilizados:
\begin{itemize}
	\item Microfone capacitivo de campo livre:
		\begin{itemize}
			\item Tipo número 4189-A-021;
			\item Sensibilidade -26.7 dB re 1V/Pa;
			\item Incerteza, 95\% de nível de confiança de 0.2 dB.
		\end{itemize}
	\item Microfone capacitivo de campo difuso:
		\begin{itemize}
			\item Tipo número 4942-A-021;
			\item Sensibilidade -26.3 dB re 1V/Pa;
			\item Incerteza, 95\% de nível de confiança de 0.2 dB.
		\end{itemize}
	\item Calibrador de microfone tipo CAL 200, da Larson Davis. Nível de calibração 94/114 dB;
	\item Fonte sonora tipo 4204:
		\begin{itemize}
			\item Cumpre ISO 3741 , ISO 3747 e ISO 6926 para calibrar fontes de potência sonora;
			\item Gama de frequências de 100 Hz a 20 kHz;
			\item Saída de potência sonora 91 dB re 1 pW (peso A, freqüência de linha de 50 Hz) e 95 dB re 1 pW (Aweighted , freqüência de linha de 60 Hz);
			\item Gama de temperaturas de -10ºC a + 50ºC;
			\item Operação 50 e 60 Hz.
		\end{itemize}
	\item Tripé e Cabos;
	\item Analisador de sinais modelo SCADAS da LMS:
	\begin{itemize}
		\item módulo de condicionamento e aquisição (com 4 ou 8 canais dinâmicos) com frequência de amostragem de 102,4 KHz e 24 bits de resolução;
		\item duas entradas para tacómetro com taxas de amostragem de até 6,5 MHz;
		\item dois geradores de função.
	\end{itemize}
	\item Rotating boom tipo 3923:
		\begin{itemize}
			\item Cumpre ISO 3741;
			\item Comprimento da lança ajustável entre 50 cm e 200 cm;
			\item Operação com bateria com células NiCd ou operação de linha embutidos;
			\item Três vezes rotação do interruptor selecionável;
			\item Plano de rotação ajustável em passos de 10 graus;
			\item E poder de sinal do microfone via anéis deslizantes;
			\item Potência sonora emitida típico igual a 26 dB re 1 pW (peso A).
		\end{itemize}
\end{itemize}
% Medição de pressão sonora na câmara semi-anecoica
% Medição de pressão sonora na câmara reverberante
%        grafico indices absorcao

% codigo

Focando a formatação dos dados dos microfones, foi desenvolvido um script python. Segue o mesmo:


Focando o processamento dos dados e plotagem dos gráficos foi desenvolvido um script em Matlab. Segue o mesmo:
%\lstinputlisting{../resultados_nossos/script_matlab_potencia_sonora.m}

\chapter{Resultados}\label{resultados}
% grafico semianecoica
Feito os procedimentos das medições, obteve-se os níveis de pressão sonora em dB por bandas de frequências que pode ser conferido na figura \ref{figura_3}.
\begin{figure}[h!]
    %\centering
    \hspace{-4.5cm}
    \includegraphics[width=1.6\textwidth]{figuras/P_anecoica.eps}
    \caption{Níveis de pressão sonora da fonte na camara semianecóica por bandas de frequências. Fonte: autoria própria.}
    \label{figura_3}
\end{figure}

Tendo em vista a figura \ref{figura_3}, é possível observar que o nível de pressão sonora para cada banda de frequência está na média de 75 dB e máximo perto de 78 dB e os ruídos mais fortes estão concentrados mais na faixa de médias frequências.


% grafico ruido de fundo semianecoica
\newpage
Obteve-se também o ruído de fundo da câmara semianecóica ilustrado na figura \ref{figura_4}.
\begin{figure}[h!]
    %\centering
    \hspace{-4.5cm}
    \includegraphics[width=1.6\textwidth]{figuras/P_anecoica.eps}
    \caption{Ruído de fundo da câmara semianecóica. Fonte: autoria própria.}
    \label{figura_4}
\end{figure}

Tendo em vista a figura \ref{figura_4}, é possível observar que o nível de pressão sonora para cada banda de frequência está na média de 14 dB e máximo perto de 20 dB e os ruídos mais fortes estão mais na faixa de baixas frequências. É visível que o ruído gravado da fonte é mais alto que o ruído de fundo.

% grafico reverberante
\newpage
Obteve-se também os níveis de pressão sonora da fonte na camara reverberante por bandas de frequências. Tais valores estão ilustrados na figura \ref{figura_5}.
\begin{figure}[h!]
    %\centering
    \hspace{-4.5cm}
    \includegraphics[width=1.6\textwidth]{figuras/P_anecoica.eps}
    \caption{Níveis de pressão sonora da fonte na camara reverberante por bandas de frequências. Fonte: autoria própria.}
    \label{figura_5}
\end{figure}

Tendo em vista a figura \ref{figura_5}, é possível observar que o nível de pressão sonora para cada banda de frequência está na média de 75 dB e máximo perto de 81 dB e os ruídos mais fortes estão concentrados mais na faixa de médias frequências. É visível que esses ruídos encontrados possuem uma correlação bastante alta com os encontrados na câmara semianecóica.

% grafico ruido de fundo reverberante
\newpage
Obteve-se também o ruído de fundo da câmara reverberante ilustrado na figura \ref{figura_6}.
\begin{figure}[h!]
     %\centering
    \hspace{-4.5cm}
    \includegraphics[width=1.6\textwidth]{figuras/P_anecoica.eps}
    \caption{Ruído de fundo da câmara reverberante. Fonte: autoria própria.}
    \label{figura_6}
\end{figure}

Tendo em vista a figura \ref{figura_6}, é possível observar que o nível de pressão sonora para cada banda de frequência está na média de 46 dB e máximo perto de 54 dB e os ruídos mais fortes estão mais na faixa de baixas frequências. 

\newpage
Também foi calculado os coeficientes de absorção da câmara reverberante e os mesmos estão ilustrados na figura \ref{figura_7}.
\begin{figure}[h!]
    %\centering
    \hspace{-4.5cm}
    \includegraphics[width=1.6\textwidth]{figuras/P_anecoica.eps}
    \caption{Índices de absorção da câmara reverberante. Fonte: autoria própria.}
    \label{figura_7}
\end{figure}
Na figura \ref{figura_7} é visível que os índices possuem a mesma caracterização para as bandas de frequências abordadas.

% comparacao potencia entre os graficos
\newpage
Como resultado principal foi calculado os níveis de potência sonora em cada sala e comparados com os níveis de referência tabelados da fonte sonora. A figura \ref{figura_8} ilustra esse processo e é possível observar que os ruídos para cada banda de frequência segue o formato tabelado de referência.
\begin{figure}[h!]
    %\centering
    \hspace{-4.5cm}
    \includegraphics[width=1.6\textwidth]{figuras/P_anecoica.eps}
    \caption{Comparação das potências sonoras. Fonte: autoria própria.}
    \label{figura_8}
\end{figure}

\chapter{Conclusões}\label{conclusoes}

Nesse presente relatório trabalhou-se com um procedimento experimental para mensurar a potência sonora nos ambientes de câmara semianecóica e câmara reverberante. Para tal fim validou-se as potências mensuradas com as tabeladas de referência da fonte sonora. As medições foram realizadas e códigos de processamento dos dados foram desenvolvidos. Observou-se no final do processo que as potências	mensuradas realmente caracterizaram a fonte em si com os valores tabelados, cabendo ressaltar que as discrepâncias foram causadas em parte pelos respectivos ruídos de fundo de cada sala. O ambiente de medição que mais se aproximou dos valores tabelados foi câmara reverberante.